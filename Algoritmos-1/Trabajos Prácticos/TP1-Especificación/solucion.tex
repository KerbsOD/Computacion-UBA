% CONFIGURACIÓN DEL PROYECTO
\documentclass[a4paper]{article}
\setlength{\parskip}{2mm}
\newcommand{\tab}{~ \qquad}
\usepackage{ifthen}
\usepackage{amssymb}
\usepackage{multicol}
\usepackage{graphicx}
\usepackage[absolute]{textpos}
\usepackage{amsmath, amscd, amssymb, amsthm, latexsym}
% \usepackage[noload]{qtree}
%\usepackage{xspace,rotating,calligra,dsfont,ifthen}
\usepackage{xspace,rotating,dsfont,ifthen}
\usepackage[spanish,activeacute]{babel}
\usepackage[utf8]{inputenc}
\usepackage{pgfpages}
\usepackage{pgf,pgfarrows,pgfnodes,pgfautomata,pgfheaps,xspace,dsfont}
\usepackage{listings}
\usepackage{multicol}
\usepackage{todonotes}
\usepackage{url}
\usepackage{float}
\usepackage{framed,mdframed}
\usepackage{cancel}

\usepackage[strict]{changepage}


\makeatletter


\newcommand\hfrac[2]{\genfrac{}{}{0pt}{}{#1}{#2}} %\hfrac{}{} es un \frac sin la linea del medio

\newcommand\Wider[2][3em]{% \Wider[3em]{} reduce los m\'argenes
\makebox[\linewidth][c]{%
  \begin{minipage}{\dimexpr\textwidth+#1\relax}
  \raggedright#2
  \end{minipage}%
  }%
}


\@ifclassloaded{beamer}{%
  \newcommand{\tocarEspacios}{%
    \addtolength{\leftskip}{4em}%
    \addtolength{\parindent}{-3em}%
  }%
}
{%
  \usepackage[top=1cm,bottom=2cm,left=1cm,right=1cm]{geometry}%
  \usepackage{color}%
  \newcommand{\tocarEspacios}{%
    \addtolength{\leftskip}{3em}%
    \setlength{\parindent}{0em}%
  }%
}

\newcommand{\encabezadoDeProc}[4]{%
  % Ponemos la palabrita problema en tt
%  \noindent%
  {\normalfont\bfseries\ttfamily proc}%
  % Ponemos el nombre del problema
  \ %
  {\normalfont\ttfamily #2}%
  \
  % Ponemos los parametros
  (#3)%
  \ifthenelse{\equal{#4}{}}{}{%
  \ =\ %
  % Ponemos el nombre del resultado
  {\normalfont\ttfamily #1}%
  % Por ultimo, va el tipo del resultado
  \ : #4}
}

\newcommand{\encabezadoDeTipo}[2]{%
  % Ponemos la palabrita tipo en tt
  {\normalfont\bfseries\ttfamily tipo}%
  % Ponemos el nombre del tipo
  \ %
  {\normalfont\ttfamily #2}%
  \ifthenelse{\equal{#1}{}}{}{$\langle$#1$\rangle$}
}

% Primero definiciones de cosas al estilo title, author, date

\def\materia#1{\gdef\@materia{#1}}
\def\@materia{No especifi\'o la materia}
\def\lamateria{\@materia}

\def\cuatrimestre#1{\gdef\@cuatrimestre{#1}}
\def\@cuatrimestre{No especifi\'o el cuatrimestre}
\def\elcuatrimestre{\@cuatrimestre}

\def\anio#1{\gdef\@anio{#1}}
\def\@anio{No especifi\'o el anio}
\def\elanio{\@anio}

\def\fecha#1{\gdef\@fecha{#1}}
\def\@fecha{\today}
\def\lafecha{\@fecha}

\def\nombre#1{\gdef\@nombre{#1}}
\def\@nombre{No especific'o el nombre}
\def\elnombre{\@nombre}

\def\practicas#1{\gdef\@practica{#1}}
\def\@practica{No especifi\'o el n\'umero de pr\'actica}
\def\lapractica{\@practica}


% Esta macro convierte el numero de cuatrimestre a palabras
\newcommand{\cuatrimestreLindo}{
  \ifthenelse{\equal{\elcuatrimestre}{1}}
  {Primer cuatrimestre}
  {\ifthenelse{\equal{\elcuatrimestre}{2}}
  {Segundo cuatrimestre}
  {Verano}}
}


\newcommand{\depto}{{UBA -- Facultad de Ciencias Exactas y Naturales --
      Departamento de Computaci\'on}}

\newcommand{\titulopractica}{
  \centerline{\depto}
  \vspace{1ex}
  \centerline{{\Large\lamateria}}
  \vspace{0.5ex}
  \centerline{\cuatrimestreLindo de \elanio}
  \vspace{2ex}
  \centerline{{\huge Pr\'actica \lapractica -- \elnombre}}
  \vspace{5ex}
  \arreglarincisos
  \newcounter{ejercicio}
  \newenvironment{ejercicio}{\stepcounter{ejercicio}\textbf{Ejercicio
      \theejercicio}%
    \renewcommand\@currentlabel{\theejercicio}%
  }{\vspace{0.2cm}}
}


\newcommand{\titulotp}{
  \centerline{\depto}
  \vspace{1ex}
  \centerline{{\Large\lamateria}}
  \vspace{0.5ex}
  \centerline{\cuatrimestreLindo de \elanio}
  \vspace{0.5ex}
  \centerline{\lafecha}
  \vspace{2ex}
  \centerline{{\huge\elnombre}}
  \vspace{5ex}
}


%practicas
\newcommand{\practica}[2]{%
    \title{Pr\'actica #1 \\ #2}
    \author{Algoritmos y Estructuras de Datos I}
    \date{Primer Cuatrimestre 2022}

    \maketitlepractica{#1}{#2}
}

\newcommand \maketitlepractica[2] {%
\begin{center}
\begin{tabular}{r cr}
 \begin{tabular}{c}
{\large\bf\textsf{\ Algoritmos y Estructuras de Datos I\ }}\\
Primer Cuatrimestre 2022\\
\title{\normalsize Gu\'ia Pr\'actica #1 \\ \textbf{#2}}\\
\@title
\end{tabular} &
\begin{tabular}{@{} p{1.6cm} @{}}
\includegraphics[width=1.6cm]{logodpt.jpg}
\end{tabular} &
\begin{tabular}{l @{}}
 \emph{Departamento de Computaci\'on} \\
 \emph{Facultad de Ciencias Exactas y Naturales} \\
 \emph{Universidad de Buenos Aires} \\
\end{tabular}
\end{tabular}
\end{center}

\bigskip
}


% Símbolos varios

\newcommand{\nat}{\ensuremath{\mathds{N}}}
\newcommand{\ent}{\ensuremath{\mathds{Z}}}
\newcommand{\float}{\ensuremath{\mathds{R}}}
\newcommand{\bool}{\ensuremath{\mathsf{Bool}}}
\newcommand{\True}{\ensuremath{\mathrm{true}}}
\newcommand{\False}{\ensuremath{\mathrm{false}}}
\newcommand{\Then}{\ensuremath{\rightarrow}}
\newcommand{\Iff}{\ensuremath{\leftrightarrow}}
\newcommand{\implica}{\ensuremath{\longrightarrow}}
\newcommand{\IfThenElse}[3]{\ensuremath{\mathsf{if}\ #1\ \mathsf{then}\ #2\ \mathsf{else}\ #3\ \mathsf{fi}}}
\newcommand{\In}{\textsf{in }}
\newcommand{\Out}{\textsf{out }}
\newcommand{\Inout}{\textsf{inout }}
\newcommand{\yLuego}{\land _L}
\newcommand{\oLuego}{\lor _L}
\newcommand{\implicaLuego}{\implica _L}
\newcommand{\cuantificador}[5]{%
	\ensuremath{(#2 #3: #4)\ (%
		\ifthenelse{\equal{#1}{unalinea}}{
			#5
		}{
			$ % exiting math mode
			\begin{adjustwidth}{+2em}{}
			$#5$%
			\end{adjustwidth}%
			$ % entering math mode
		}
	)}
}

\newcommand{\existe}[4][]{%
	\cuantificador{#1}{\exists}{#2}{#3}{#4}
}
\newcommand{\paraTodo}[4][]{%
	\cuantificador{#1}{\forall}{#2}{#3}{#4}
}

% Símbolo para marcar los ejercicios importantes (estrellita)
\newcommand\importante{\raisebox{0.5pt}{\ensuremath{\bigstar}}}


\newcommand{\rango}[2]{[#1\twodots#2]}
\newcommand{\comp}[2]{[\,#1\,|\,#2\,]}

\newcommand{\rangoac}[2]{(#1\twodots#2]}
\newcommand{\rangoca}[2]{[#1\twodots#2)}
\newcommand{\rangoaa}[2]{(#1\twodots#2)}

%ejercicios
\newtheorem{exercise}{Ejercicio}
\newenvironment{ejercicio}[1][]{\begin{exercise}#1\rm}{\end{exercise} \vspace{0.2cm}}
\newenvironment{items}{\begin{enumerate}[a)]}{\end{enumerate}}
\newenvironment{subitems}{\begin{enumerate}[i)]}{\end{enumerate}}
\newcommand{\sugerencia}[1]{\noindent \textbf{Sugerencia:} #1}

\lstnewenvironment{code}{
    \lstset{% general command to set parameter(s)
        language=C++, basicstyle=\small\ttfamily, keywordstyle=\slshape,
        emph=[1]{tipo,usa}, emphstyle={[1]\sffamily\bfseries},
        morekeywords={tint,forn,forsn},
        basewidth={0.47em,0.40em},
        columns=fixed, fontadjust, resetmargins, xrightmargin=5pt, xleftmargin=15pt,
        flexiblecolumns=false, tabsize=2, breaklines, breakatwhitespace=false, extendedchars=true,
        numbers=left, numberstyle=\tiny, stepnumber=1, numbersep=9pt,
        frame=l, framesep=3pt,
    }
   \csname lst@SetFirstLabel\endcsname}
  {\csname lst@SaveFirstLabel\endcsname}


%tipos basicos
\newcommand{\rea}{\ensuremath{\mathsf{Float}}}
\newcommand{\cha}{\ensuremath{\mathsf{Char}}}
\newcommand{\str}{\ensuremath{\mathsf{String}}}

\newcommand{\mcd}{\mathrm{mcd}}
\newcommand{\prm}[1]{\ensuremath{\mathsf{prm}(#1)}}
\newcommand{\sgd}[1]{\ensuremath{\mathsf{sgd}(#1)}}

\newcommand{\tuple}[2]{\ensuremath{#1 \times #2}}

%listas
\newcommand{\TLista}[1]{\ensuremath{seq \langle #1\rangle}}
\newcommand{\lvacia}{\ensuremath{[\ ]}}
\newcommand{\lv}{\ensuremath{[\ ]}}
\newcommand{\longitud}[1]{\ensuremath{|#1|}}
\newcommand{\cons}[1]{\ensuremath{\mathsf{addFirst}}(#1)}
\newcommand{\indice}[1]{\ensuremath{\mathsf{indice}}(#1)}
\newcommand{\conc}[1]{\ensuremath{\mathsf{concat}}(#1)}
\newcommand{\cab}[1]{\ensuremath{\mathsf{head}}(#1)}
\newcommand{\cola}[1]{\ensuremath{\mathsf{tail}}(#1)}
\newcommand{\sub}[1]{\ensuremath{\mathsf{subseq}}(#1)}
\newcommand{\en}[1]{\ensuremath{\mathsf{en}}(#1)}
\newcommand{\cuenta}[2]{\mathsf{cuenta}\ensuremath{(#1, #2)}}
\newcommand{\suma}[1]{\mathsf{suma}(#1)}
\newcommand{\twodots}{\ensuremath{\mathrm{..}}}
\newcommand{\masmas}{\ensuremath{++}}
\newcommand{\matriz}[1]{\TLista{\TLista{#1}}}

\newcommand{\seqchar}{\TLista{\cha}}


% Acumulador
\newcommand{\acum}[1]{\ensuremath{\mathsf{acum}}(#1)}
\newcommand{\acumselec}[3]{\ensuremath{\mathrm{acum}(#1 |  #2, #3)}}

% \selector{variable}{dominio}
\newcommand{\selector}[2]{#1~\ensuremath{\leftarrow}~#2}
\newcommand{\selec}{\ensuremath{\leftarrow}}

\newcommand{\pred}[3]{%
    {\normalfont\bfseries\ttfamily\noindent pred }%
    {\normalfont\ttfamily #1}%
    \ifthenelse{\equal{#2}{}}{}{\ (#2) }%
    \{%
    \begin{adjustwidth}{+2em}{}
      \ensuremath{#3}
    \end{adjustwidth}
    \}%
    {\normalfont\bfseries\,\par}%
}

\newenvironment{proc}[4][res]{%

  % El parametro 1 (opcional) es el nombre del resultado
  % El parametro 2 es el nombre del problema
  % El parametro 3 son los parametros
  % El parametro 4 es el tipo del resultado
  % Preambulo del ambiente problema
  % Tenemos que definir los comandos requiere, asegura, modifica y aux
  \newcommand{\pre}[2][]{%
    {\normalfont\bfseries\ttfamily Pre}%
    \ifthenelse{\equal{##1}{}}{}{\ {\normalfont\ttfamily ##1} :}\ %
    \{\ensuremath{##2}\}%
    {\normalfont\bfseries\,\par}%
  }
  \newcommand{\post}[2][]{%
    {\normalfont\bfseries\ttfamily Post}%
    \ifthenelse{\equal{##1}{}}{}{\ {\normalfont\ttfamily ##1} :}\
    \{\ensuremath{##2}\}%
    {\normalfont\bfseries\,\par}%
  }
  \renewcommand{\aux}[4]{%
    {\normalfont\bfseries\ttfamily aux\ }%
    {\normalfont\ttfamily ##1}%
    \ifthenelse{\equal{##2}{}}{}{\ (##2)}\ : ##3\, = \ensuremath{##4}%
    {\normalfont\bfseries\,;\par}%
  }
  \renewcommand{\pred}[3]{%
    {\normalfont\bfseries\ttfamily pred }%
    {\normalfont\ttfamily ##1}%
    \ifthenelse{\equal{##2}{}}{}{\ (##2) }%
    \{%
    \begin{adjustwidth}{+5em}{}
      \ensuremath{##3}
    \end{adjustwidth}
    \}%
    {\normalfont\bfseries\,\par}%
  }

  \newcommand{\res}{#1}
  \vspace{1ex}
  \noindent
  \encabezadoDeProc{#1}{#2}{#3}{#4}
  % Abrimos la llave
  \{\par%
  \tocarEspacios
}
% Ahora viene el cierre del ambiente problema
{
  % Cerramos la llave
  \noindent\}
  \vspace{1ex}
}


\newcommand{\aux}[4]{%
    {\normalfont\bfseries\ttfamily\noindent aux\ }%
    {\normalfont\ttfamily #1}%
    \ifthenelse{\equal{#2}{}}{}{\ (#2)}\ : #3\, = \ensuremath{#4}%
    {\normalfont\bfseries\,;\par}%
}


% \newcommand{\pre}[1]{\textsf{pre}\ensuremath{(#1)}}

\newcommand{\procnom}[1]{\textsf{#1}}
\newcommand{\procil}[3]{\textsf{proc #1}\ensuremath{(#2) = #3}}
\newcommand{\procilsinres}[2]{\textsf{proc #1}\ensuremath{(#2)}}
\newcommand{\preil}[2]{\textsf{Pre #1: }\ensuremath{#2}}
\newcommand{\postil}[2]{\textsf{Post #1: }\ensuremath{#2}}
\newcommand{\auxil}[2]{\textsf{fun }\ensuremath{#1 = #2}}
\newcommand{\auxilc}[4]{\textsf{fun }\ensuremath{#1( #2 ): #3 = #4}}
\newcommand{\auxnom}[1]{\textsf{fun }\ensuremath{#1}}
\newcommand{\auxpred}[3]{\textsf{pred }\ensuremath{#1( #2 ) \{ #3 \}}}

\newcommand{\comentario}[1]{{/*\ #1\ */}}

\newcommand{\nom}[1]{\ensuremath{\mathsf{#1}}}


% En las practicas/parciales usamos numeros arabigos para los ejercicios.
% Aca cambiamos los enumerate comunes para que usen letras y numeros
% romanos
\newcommand{\arreglarincisos}{%
  \renewcommand{\theenumi}{\alph{enumi}}
  \renewcommand{\theenumii}{\roman{enumii}}
  \renewcommand{\labelenumi}{\theenumi)}
  \renewcommand{\labelenumii}{\theenumii)}
}



%%%%%%%%%%%%%%%%%%%%%%%%%%%%%% PARCIAL %%%%%%%%%%%%%%%%%%%%%%%%
\let\@xa\expandafter
\newcommand{\tituloparcial}{\centerline{\depto -- \lamateria}
  \centerline{\elnombre -- \lafecha}%
  \setlength{\TPHorizModule}{10mm} % Fija las unidades de textpos
  \setlength{\TPVertModule}{\TPHorizModule} % Fija las unidades de
                                % textpos
  \arreglarincisos
  \newcounter{total}% Este contador va a guardar cuantos incisos hay
                    % en el parcial. Si un ejercicio no tiene incisos,
                    % cuenta como un inciso.
  \newcounter{contgrilla} % Para hacer ciclos
  \newcounter{columnainicial} % Se van a usar para los cline cuando un
  \newcounter{columnafinal}   % ejercicio tenga incisos.
  \newcommand{\primerafila}{}
  \newcommand{\segundafila}{}
  \newcommand{\rayitas}{} % Esto va a guardar los \cline de los
                          % ejercicios con incisos, asi queda mas bonito
  \newcommand{\anchodegrilla}{20} % Es para textpos
  \newcommand{\izquierda}{7} % Estos dos le dicen a textpos donde colocar
  \newcommand{\abajo}{2}     % la grilla
  \newcommand{\anchodecasilla}{0.4cm}
  \setcounter{columnainicial}{1}
  \setcounter{total}{0}
  \newcounter{ejercicio}
  \setcounter{ejercicio}{0}
  \renewenvironment{ejercicio}[1]
  {%
    \stepcounter{ejercicio}\textbf{\noindent Ejercicio \theejercicio. [##1
      puntos]}% Formato
    \renewcommand\@currentlabel{\theejercicio}% Esto es para las
                                % referencias
    \newcommand{\invariante}[2]{%
      {\normalfont\bfseries\ttfamily invariante}%
      \ ####1\hspace{1em}####2%
    }%
    \newcommand{\Proc}[5][result]{
      \encabezadoDeProc{####1}{####2}{####3}{####4}\hspace{1em}####5}%
  }% Aca se termina el principio del ejercicio
  {% Ahora viene el final
    % Esto suma la cantidad de incisos o 1 si no hubo ninguno
    \ifthenelse{\equal{\value{enumi}}{0}}
    {\addtocounter{total}{1}}
    {\addtocounter{total}{\value{enumi}}}
    \ifthenelse{\equal{\value{ejercicio}}{1}}{}
    {
      \g@addto@macro\primerafila{&} % Si no estoy en el primer ej.
      \g@addto@macro\segundafila{&}
    }
    \ifthenelse{\equal{\value{enumi}}{0}}
    {% No tiene incisos
      \g@addto@macro\primerafila{\multicolumn{1}{|c|}}
      \bgroup% avoid overwriting somebody else's value of \tmp@a
      \protected@edef\tmp@a{\theejercicio}% expand as far as we can
      \@xa\g@addto@macro\@xa\primerafila\@xa{\tmp@a}%
      \egroup% restore old value of \tmp@a, effect of \g@addto.. is

      \stepcounter{columnainicial}
    }
    {% Tiene incisos
      % Primero ponemos el encabezado
      \g@addto@macro\primerafila{\multicolumn}% Ahora el numero de items
      \bgroup% avoid overwriting somebody else's value of \tmp@a
      \protected@edef\tmp@a{\arabic{enumi}}% expand as far as we can
      \@xa\g@addto@macro\@xa\primerafila\@xa{\tmp@a}%
      \egroup% restore old value of \tmp@a, effect of \g@addto.. is
      % global
      % Ahora el formato
      \g@addto@macro\primerafila{{|c|}}%
      % Ahora el numero de ejercicio
      \bgroup% avoid overwriting somebody else's value of \tmp@a
      \protected@edef\tmp@a{\theejercicio}% expand as far as we can
      \@xa\g@addto@macro\@xa\primerafila\@xa{\tmp@a}%
      \egroup% restore old value of \tmp@a, effect of \g@addto.. is
      % global
      % Ahora armamos la segunda fila
      \g@addto@macro\segundafila{\multicolumn{1}{|c|}{a}}%
      \setcounter{contgrilla}{1}
      \whiledo{\value{contgrilla}<\value{enumi}}
      {%
        \stepcounter{contgrilla}
        \g@addto@macro\segundafila{&\multicolumn{1}{|c|}}
        \bgroup% avoid overwriting somebody else's value of \tmp@a
        \protected@edef\tmp@a{\alph{contgrilla}}% expand as far as we can
        \@xa\g@addto@macro\@xa\segundafila\@xa{\tmp@a}%
        \egroup% restore old value of \tmp@a, effect of \g@addto.. is
        % global
      }
      % Ahora armo las rayitas
      \setcounter{columnafinal}{\value{columnainicial}}
      \addtocounter{columnafinal}{-1}
      \addtocounter{columnafinal}{\value{enumi}}
      \bgroup% avoid overwriting somebody else's value of \tmp@a
      \protected@edef\tmp@a{\noexpand\cline{%
          \thecolumnainicial-\thecolumnafinal}}%
      \@xa\g@addto@macro\@xa\rayitas\@xa{\tmp@a}%
      \egroup% restore old value of \tmp@a, effect of \g@addto.. is
      \setcounter{columnainicial}{\value{columnafinal}}
      \stepcounter{columnainicial}
    }
    \setcounter{enumi}{0}%
    \vspace{0.2cm}%
  }%
  \newcommand{\tercerafila}{}
  \newcommand{\armartercerafila}{
    \setcounter{contgrilla}{1}
    \whiledo{\value{contgrilla}<\value{total}}
    {\stepcounter{contgrilla}\g@addto@macro\tercerafila{&}}
  }
  \newcommand{\grilla}{%
    \g@addto@macro\primerafila{&\textbf{TOTAL}}
    \g@addto@macro\segundafila{&}
    \g@addto@macro\tercerafila{&}
    \armartercerafila
    \ifthenelse{\equal{\value{total}}{\value{ejercicio}}}
    {% No hubo incisos
      \begin{textblock}{\anchodegrilla}(\izquierda,\abajo)
        \begin{tabular}{|*{\value{total}}{p{\anchodecasilla}|}c|}
          \hline
          \primerafila\\
          \hline
          \tercerafila\\
          \tercerafila\\
          \hline
        \end{tabular}
      \end{textblock}
    }
    {% Hubo incisos
      \begin{textblock}{\anchodegrilla}(\izquierda,\abajo)
        \begin{tabular}{|*{\value{total}}{p{\anchodecasilla}|}c|}
          \hline
          \primerafila\\
          \rayitas
          \segundafila\\
          \hline
          \tercerafila\\
          \tercerafila\\
          \hline
        \end{tabular}
      \end{textblock}
    }
  }%
  % \datosalumno
}

\newcommand{\datosalumno}{
  \vspace{0.4cm}
  \textbf{Apellidos:}

  \textbf{Nombres:}

  \textbf{LU:}

  \textbf{Correo electrónico:}

  \textbf{Nro. de carillas que adjunta:}
  \vspace{0.5cm}
}


% AMBIENTE CONSIGNAS
% Se usa en el TP para ir agregando las cosas que tienen que resolver
% los alumnos.
% Dentro del ambiente hay que usar \item para cada consigna

\newcounter{consigna}
\setcounter{consigna}{0}

\newenvironment{consignas}{%
  \newcommand{\consigna}{\stepcounter{consigna}\textbf{\theconsigna.}}%
  \renewcommand{\ejercicio}[1]{\item ##1 }
  \renewcommand{\proc}[5][result]{\item
    \encabezadoDeProc{##1}{##2}{##3}{##4}\hspace{1em}##5}%
  \newcommand{\invariante}[2]{\item%
    {\normalfont\bfseries\ttfamily invariante}%
    \ ##1\hspace{1em}##2%
  }
  \renewcommand{\aux}[4]{\item%
    {\normalfont\bfseries\ttfamily aux\ }%
    {\normalfont\ttfamily ##1}%
    \ifthenelse{\equal{##2}{}}{}{\ (##2)}\ : ##3 \hspace{1em}##4%
  }
  % Comienza la lista de consignas
  \begin{list}{\consigna}{%
      \setlength{\itemsep}{0.5em}%
      \setlength{\parsep}{0cm}%
    }
}%
{\end{list}}



% para decidir si usar && o ^
\newcommand{\y}[0]{\ensuremath{\land}}

% macros de correctitud
\newcommand{\semanticComment}[2]{#1 \ensuremath{#2};}
\newcommand{\namedSemanticComment}[3]{#1 #2: \ensuremath{#3};}


\newcommand{\local}[1]{\semanticComment{local}{#1}}

\newcommand{\vale}[1]{\semanticComment{vale}{#1}}
\newcommand{\valeN}[2]{\namedSemanticComment{vale}{#1}{#2}}
\newcommand{\impl}[1]{\semanticComment{implica}{#1}}
\newcommand{\implN}[2]{\namedSemanticComment{implica}{#1}{#2}}
\newcommand{\estado}[1]{\semanticComment{estado}{#1}}

\newcommand{\invarianteCN}[2]{\namedSemanticComment{invariante}{#1}{#2}}
\newcommand{\invarianteC}[1]{\semanticComment{invariante}{#1}}
\newcommand{\varianteCN}[2]{\namedSemanticComment{variante}{#1}{#2}}
\newcommand{\varianteC}[1]{\semanticComment{variante}{#1}}

\usepackage{caratula} 
\usepackage{hyperref}
\usepackage{scrextend}
\usepackage{blindtext}




% CARÁTULA
\begin{document}

\titulo{TP de Especificación}
\subtitulo{Trabajo Práctico Grupal}
\fecha{30 de Marzo de 2022}
\materia{Algoritmos y Estructuras de Datos I}
\grupo{Grupo 10}

\newcommand{\dato}{\textit{Dato}}
\newcommand{\individuo}{\textit{Individuo}}

\integrante{Dominguez, Emilia}{37752993}{maemiliadominguez@gmail.com}
\integrante{Kerbs, Octavio}{64/22}{octaviokerbs@gmail.com}
\integrante{Russo, Gabriel}{107/19}{gabrielrussoguiot@gmail.com}
\integrante{Traverso, Lucas}{479/18}{lucas6246@gmail.com}

\maketitle


% ÍNDICE
\tableofcontents
\newpage


% DEFINICIÓN DE TIPOS
\section{Definición de Tipos}
\begin{description}
	\item type \textit{pos} = \(\ent \times \ent\) 
	\item type \textit{tablero} = \( \TLista{\TLista{\bool}}\) 
	\item type \textit{jugadas} = \(\TLista{\textit{pos} \times \ent}\) 
	\item type \textit{banderitas} = \(\TLista{\textit{pos}}\) 
\end{description}
\newpage


% FUNCIONES AUXILIARES--------------------------------------------------


% PROBLEMAS % 
\section{Problemas}

% PARTE I-----------------------------------------------%
\subsection{Parte I: Juego b\'asico}


% -------------------------------------------EJERCICIO 1-------------------------------------------%
\subsubsection{Ejercicio 1}
\begin{addmargin}[4em]{1em}
\aux{minasAdyacentes}{t: \textit{tablero}, p: \textit{pos}}{\ent}{
	\newline
	\sum_{i = max(p[0] - 1,\ 0)}^{min(p[0] + 1,\ \longitud{t} - 1)} \sum_{j = max(p[1] - 1,\ 0)}^{min(p[1] + 1,\ \longitud{t} - 1)}
	\ es1SiPosicionEsBombaSino0(t,\ i,\ j)\ -
	\newline es1SiPosicionEsBombaSino0(t,\ p[0],\ p[1])}
\end{addmargin}





% -------------------------------------------EJERCICIO 2-------------------------------------------%
\subsubsection{Ejercicio 2}
\begin{addmargin}[4em]{1em}
\pred{juegoValido}{t: \textit{tablero}, j: \textit{jugadas}}{
	(tableroValido(t)\ \wedge \newline
	todasLasPosicionesDeLaJugadaPertenecenAlTablero(t, j)\ \wedge \newline
	noExistenPosicionesRepetidasEnLaJugada(j))\ \yLuego \newline
	(esLaCantidadDeMinasAdyacentesCorrectaParaTodaLaJugada(t,\ j)\ \wedge\ 
	\newline cantidadDeBombasEnPosicionesDeLaJugada(t,\ j)\ \leq 1)
}
\end{addmargin}




% -------------------------------------------EJERCICIO 3-------------------------------------------%
\subsubsection{Ejercicio 3}
\begin{addmargin}[4em]{1em}
\begin{proc}{plantarBanderita}{in t: \textit{tablero} , in j: \textit{jugadas}, in p: \textit{pos}, inout b: \textit{banderitas}}{}{
	\pre{juegoValido(t,\ j) \wedge\ \newline 
	posicionPerteneceAlTablero(t,\ p)\ \wedge \newline 
	\neg\ posicionPerteneceAJugadas(p,\ j)\ \wedge \newline 
	\neg\ posicionPerteneceABanderitas(p,\ b)\ \wedge \newline 
	banderitasValidasParaLaJugada(b,\ j,\ t)\ \wedge \newline 
	b = b_{0}}
	\post{posicionPerteneceABanderitas(p,\ b)\ \wedge \newline
	todasLasPosicionesDeB_{1}PertenecenAB_{2}(b_{0},\ b)\ \wedge \newline 
	(\longitud{b} = \longitud{b_{0}}\ +\ 1)}
}
\end{proc}
\end{addmargin}




% -------------------------------------------EJERCICIO 4-------------------------------------------%
\subsubsection{Ejercicio 4}
\begin{addmargin}[4em]{1em}
\begin{proc}{perdi\'o}{in t: \textit{tablero} , in j: \textit{jugadas}, out res: $\bool$}{}{
	\pre{juegoValido(t,\ j)}
	\post{res = true \iff\ cantidadDeBombasEnPosicionesDeLaJugada(t,\ j) = 1}
}
\end{proc}
\end{addmargin}


% -------------------------------------------EJERCICIO 5-------------------------------------------%
\subsubsection{Ejercicio 5}
\begin{addmargin}[4em]{1em}
\begin{proc}{gan\'o}{in t: \textit{tablero} , in j: \textit{jugadas}, out res: $\bool$}{}
	\pre{juegoValido(t,\ j)}
	\post{res = true \iff\ cantidadDeBombasEnPosicionesDeLaJugada(t,\ j) = 0\ \wedge \newline
	jugadasTodasLasPosicionesSinBombas(t,\ j)}
\end{proc}
\end{addmargin}

\bigbreak
% -------------------------------------------EJERCICIO 6-------------------------------------------%
\subsubsection{Ejercicio 6}
\begin{addmargin}[4em]{1em}
\begin{proc}{jugar}{in t: \textit{tablero} , in b: \textit{banderitas}, in p: \textit{pos}, inout j: \textit{jugadas}}{}{
	\pre{juegoValido(t,\ j) \wedge\ \newline 
	posicionPerteneceAlTablero(t,\ p)\ \wedge \newline 
	\neg\ posicionPerteneceAJugadas(p,\ j)\ \wedge \newline 
	\neg\ posicionPerteneceABanderitas(p,\ b)\ \wedge \newline 
	banderitasValidasParaLaJugada(b,\ j,\ t)\ \yLuego \newline 
	(juegoEnMarcha(j,\ t)\ \wedge\ \newline
	j = j_{0})}
	\post{posicionPerteneceAJugadas(p,\ j)\ \wedge\ \newline
	todasLasJ_{1}PertenecenAJ_{2}(j_{0},\ j)\ \wedge\ \newline
	(\longitud{j}\ = \longitud{j_{0}}\ +\ 1)}
}
\end{proc}
\end{addmargin}


% PARTE II----------------------------------------------%
\newpage
\subsection{Parte II: Despejar los vac\'ios}
% -------------------------------------------EJERCICIO 7-------------------------------------------%
\subsubsection{Ejercicio 7}
\begin{addmargin}[4em]{1em}
\pred{caminoLibre}{t: \textit{tablero}, \(p_{0}\): \textit{pos}, \(p_{1}\): \textit{pos}}{
	(\exists s:\ seq<pos>)(caminoLibreSinDefinirMinasAdyacentesALaUltimaPosicion(t,\ p_{0},\ p_{1},\ s)\ \yLuego\ \newline
	8 > minasAdyacentes(t,\ p_{1}) \geq 1)
}
\end{addmargin}

% -------------------------------------------EJERCICIO 8-------------------------------------------%
\subsubsection{Ejercicio 8}
\begin{addmargin}[4em]{1em}
\begin{proc}{jugarPlus}{in t: \textit{tablero} , in b: \textit{banderitas},
	                    in p: \textit{pos},
	                    inout j: \textit{jugadas}}{}{ 
	\pre{juegoValido(t,\ j) \wedge\ \newline 
	posicionPerteneceAlTablero(t,\ p)\ \wedge \newline 
	\neg\ posicionPerteneceAJugadas(p,\ j)\ \wedge \newline 
	\neg\ posicionPerteneceABanderitas(p,\ b)\ \wedge \newline 
	banderitasValidasParaLaJugada(b,\ j,\ t)\ \yLuego \newline 
	(juegoEnMarcha(j,\ t)\ \wedge\ \newline
	jugadasExtendidasValidas (t,\ j,\ b)\ \wedge\ \newline
	j = j_{0})}
	\post{(posicionPerteneceAJugadas(p,\ j)\ \wedge \newline
	todasLasJ_{1}PertenecenAJ_{2}(j_{0},\ j)\ \wedge\ \newline
	juegoValido(t,\ j)) \yLuego\ \newline
	(jugadasExtendidasValidas (t,\ j,\ b)\ \wedge\ \newline
	(\forall q: pos)(posicionPerteneceAlTablero(t,\ q)\ \wedge \newline 
	\neg\ posicionPerteneceAJugadas(q,\ j_{0})\ \wedge \newline 
	q \neq\ p \wedge \newline 
	\neg\ (\exists s: \TLista{pos})(caminoLibreSinDefinirMinasAdyacentesALaUltimaPosicionConBanderitas(t,\ p,\ q,\ s,\ b)) \implica\
	\neg\ posicionPerteneceAJugadas(q,\ j)))}
}

\end{proc}
\end{addmargin}


% PARTE III---------------------------------------------%
\newpage
\subsection{Parte III: Jugador autom\'atico}
% -------------------------------------------EJERCICIO 9-------------------------------------------%
\subsubsection{Ejercicio 9}
\begin{addmargin}[4em]{1em}
\begin{proc}{sugerirAutom\'atico121}{in t: \textit{tablero} , in b: \textit{banderitas},
	                                 out p: \textit{pos}}{}{ 

	\pre{juegoValido(t,\ j)\ \wedge\ juegoEnMarcha(j,\ t)\ \wedge \newline
	hayPatron121(t,\ j)}
	\post{posicionPerteneceAlTablero(p, t)\ \yLuego \newline 
	posicionNoPerteneceAJugadas(p, j) \wedge \newline
	(\exists\ s: \TLista{pos})(esPatron121(t, j, s)\ \yLuego\ sonAdyacentesNoDiagonales(p, s[1]))
	}
}

\end{proc}
\end{addmargin}
\newpage



\section{Funciones Auxiliares Y Predicados}



\subsection{Ejercicio 1}
\aux{es1SiPosicionEsBombaSino0}{t: \textit{tablero}, x, y: \ent}{\ent}{if(t[x][y]=true)\ then\ 1\ else\ 0\ fi;}
\aux{max}{x,y: \ent}{\ent}{ if(x < y)\ then\ y\ else\ x}
\aux{min}{x,y: \ent}{\ent}{ if(x < y)\ then\ x\ else\ y}



\subsection{Ejercicio 2}
\bigbreak
\pred{tableroValido}{t: \textit{tablero}}{
	(\forall i: \ent)(0\ \leq i\ <\ \longitud{t} \implicaLuego \longitud{t}\ =\ \longitud{t[i]})\ \yLuego\ (cantidadTotalDeBombas(t)\ >\ 0)
}
\bigbreak
\aux{cantidadTotalDeBombas}{t: \textit{tablero}}{\ent}{\sum_{i = 0}^{\longitud{t}-1} \sum_{j = 0}^{\longitud{t}-1}\ es1SiPosicionEsBombaSino0(\ t,\ i,\ j)}
\bigbreak
\pred{todasLasPosicionesDeLaJugadaPertenecenAlTablero}{t: \textit{tablero}, j: \textit{jugada}}{
	(\forall i:\ent)(0\ \leq\ i\ <\ \longitud{j}\ \implicaLuego\ posicionPerteneceAlTablero(t, j[i][0]))
}
\bigbreak
\pred{posicionPerteneceAlTablero}{p: \textit{pos}, t: \textit{tablero}}{
	(0\ \leq \ p_{0}\ < \longitud{t})\ \wedge\ ( 0\ \leq\ p_{1}\ < \longitud{t})
}
\bigbreak
\pred{noExistenPosicionesRepetidasEnLaJugada}{j: \textit{jugadas}}{
	(\forall i: \ent)(\forall k: \ent)((0\ \leq\ i <\ \longitud{j}\ \wedge\ 0\ \leq\ k\ <\ \longitud{j}\ \wedge\ i \neq k) \newline
	\implicaLuego\ j[i][0]\ \neq\ j[k][0])
}
\bigbreak
\pred{esLaCantidadDeMinasAdyacentesCorrectaParaTodaLaJugada}{t: \textit{tablero}, j: \textit{jugadas}}{
	(\forall i: \ent)(0\ \leq\ i\ <\ \longitud{j}\ \implicaLuego\ j[i][1]\ =\ minasAdyacentes(t, j[i][0]))
}



\subsection{Ejercicio 3}

\bigbreak
\pred{posicionPerteneceAJugadas}{p: \textit{pos}, j: \textit{jugada}}{
	(\exists i: \ent)(0\ \leq\ i\ <\  \longitud{j}\ \yLuego\ j[i][0]\ =\ p)
}
\bigbreak
\pred{posicionPerteneceABanderitas}{p: \textit{posicion}, b: \textit{banderita}}{
	(\exists i: \ent)(0\ \leq\ i\ <\  \longitud{b}\ \yLuego\ b[i]\ =\ p)
}
\bigbreak
\pred{banderitasValidasParaLaJugada}{b: \textit{banderitas}, j: \textit{jugadas}, t: \textit{tablero}}{
	noExistenPosicionesRepetidasEnLasBanderitas(b)\ \wedge\ \newline
	todasLasPosicionesDeLasBanderitasPertenecenAlTablero(t,\ b)\ \wedge \newline
	ningunaPosicionDeLaJugadaEstaEnBanderitas(j,\ b)
}
\bigbreak
\pred{noExistenPosicionesRepetidasEnLasBanderitas}{b: \textit{banderitas}}{
	(\forall i: \ent)(\forall k: \ent)((0\ \leq\ i <\ \longitud{b}\ \wedge\ 0\ \leq\ k\ <\ \longitud{b}\ \wedge\ i \neq k) \newline
	\implicaLuego\ b[i]\ \neq\ b[k])
}
\bigbreak
\pred{todasLasPosicionesDeLasBanderitasPertenecenAlTablero}{t: \textit{tablero}, b: \textit{banderitas}}{
	(\forall i: \ent)(0\ \leq\ i\ <\ \longitud{b}\ \implicaLuego\ posicionPerteneceAlTablero(t,\ b[i]))
}
\bigbreak
\pred{ningunaPosicionDeLaJugadaEstaEnBanderitas}{j: \textit{jugadas}, b: \textit{banderitas}}{
	(\forall i: \ent)(\forall k: \ent)((0\ \leq\ i\ <\ \longitud{j}\ \wedge\ 0\ \leq\ k\ <\ \longitud{b})\ \implicaLuego\ j[i][0]\ \neq\ b[k])
}
\bigbreak
\pred{\(todasLasPosicionesDeB_{1}PertenecenAB_{2}\)}{\(b_{1}\), \(b_{2}\): \textit{banderitas}}{
	(\forall x: pos)(posicionPerteneceABanderitas(x, b_{1})\ \implica\ posicionPerteneceABanderitas(x, b_{2}))
}



\subsection{Ejercicio 4}
\aux{cantidadDeBombasEnPosicionesDeLaJugada}{t: \textit{tablero}, j: \textit{jugada}}{\ent}{
	\newline \sum^{\longitud{j}-1}_{i=0}\ es1SiPosicionEsBombaSino0(\ t,\ j[i][0][0],\ j[i][0][1])}



\subsection{Ejercicio 5}
\bigbreak
\pred{jugadasTodasLasPosicionesSinBombas}{t: \textit{tablero}, j: \textit{jugadas}}{
	\longitud{j}\ =\ posicionesSinMinas(t)
}
\aux{posicionesSinMinas}{t: \textit{tablero}}{\ent}{
	(\sum_{i = 0}^{\longitud{t}-1} \sum_{k = 0}^{\longitud{t}-1} if(t[i][k]\ =\ false)\ then\ 1\ else\ 0\ fi;)}



\subsection{Ejercicio 6}
\bigbreak
\pred{juegoEnMarcha}{j: \textit{jugadas}, t: \textit{tablero}}{
	cantidadDeBombasEnPosicionesDeLaJugada(t,\ j)\ =\ 0\ \wedge\ \neg\ jugadasTodasLasPosicionesSinBombas(t,\ j)
}
\bigbreak
\pred{\(todasLasJ_{1}PertenecenAJ_{2}\)}{\(j_{1}\), \(j_{2}\): \textit{jugadas}}{
	(\forall x: pos)(posicionPerteneceAJugadas(x, j_{1})\ \implica\ posicionPerteneceAJugadas(x, j_{2}))
}



\subsection{Ejercicio 7}
\bigbreak
\pred{caminoLibreSinDefinirMinasAdyacentesALaUltimaPosicionConBanderitas}{t: \textit{tablero}, \(p_{0}\): \textit{pos}, \(p_{1}\): \textit{pos},\ s: \TLista{pos},\ b: \textit{banderitas}}{
	(posicionPerteneceASecuencia(p_{0},\ s)\ \wedge\ \newline
	posicionPerteneceASecuencia(p_{1},\ s)\ \wedge\ \newline
	(\forall p: pos)(posicionPerteneceASecuencia(p,\ s) \implica posicionPerteneceAlTablero(t,\ p)))\ \yLuego\ \newline
	((\forall p: pos)(posicionPerteneceASecuencia(p,\ s)\ \wedge\ p \neq p_{1} \implica minasAdyacentes(t,\ p) = 0)\ \wedge\ \newline 
	(\exists s_{2}:\ seq<pos>)(ningunaPosicionDeLaSecuenciaEstaEnBanderitas(s_{2},\ b)\ \wedge\ \newline 
	secuenciaDePosicionesAdyacentes(p_{0}, p_{1}, s_{2})\ \wedge\ esPermutacion(s, s_{2})))
}
\bigbreak
\pred{ningunaPosicionDeLaSecuenciaEstaEnBanderitas}{s: \TLista{pos},\ b: \textit{banderitas}}{
	(\forall i: \ent)(\forall k: \ent)((0\ \leq\ i\ <\ \longitud{b}\ \wedge\ 0\ \leq\ k\ <\ \longitud{s})\ \implicaLuego\ s[k]\ \neq\ b[i])
}
\bigbreak
\pred{posicionPerteneceASecuencia}{p: \textit{pos}, s: \TLista{pos}}{
	(\exists i:\ent)(0\ \leq\ i\ < \longitud{s}\ \yLuego\ s[i]\ =\ p)
}
\bigbreak
\pred{secuenciaDePosicionesAdyacentes}{\(p_{1}\), \(p_{2}\): \textit{pos}, s: \TLista{pos}}{
	(s[0]\ =\ p[1]\ \wedge\ s[\longitud{s}\ -\ 1]\ =\ p[2])\ \wedge \newline
	(\forall i: \ent)(0\ \leq\ i\ <\ \longitud{s}\ -\ 1\ \implicaLuego\ esAdyacente(p[i],\ p[i+1]))
}



\subsection{Ejercicio 8}
\bigbreak
\pred{jugadasExtendidasValidas}{t: \textit{tablero}, j: \textit{jugadas}, b: \textit{banderitas}}{
	(\forall p: \textit{pos})(posicionPerteneceAJugadas(p,\ j)\ \implicaLuego\ \newline 
	((0\ =\ minasAdyacentes(t,\ p) \wedge\ \newline
	todasLasPosicionesConMinasAdyacentesYCaminoLibreAPertenecenAJugada(p,\ j,\ b)) \lor \newline 
	(0\ \neq\ minasAdyacentes(t,\ p) \wedge\ \newline
	todasLasPosicionesSinMinasAdyacentesConUnaPosicionConfirmadaEnJugadaYCaminoLibreAPertenecenAJugada(p,\ j,\ b))))
}
\bigbreak
\pred{todasLasPosicionesConMinasAdyacentesYCaminoLibreAPertenecenAJugada}{t: \textit{tablero},\ p: \textit{posicion},\ j: \textit{jugada},\ b: \textit{banderitas}}{
	(\forall q: pos)(((posicionPerteneceAlTablero(t,\ q) \wedge \newline
	\neg\ posicionPerteneceABanderitas(q,\ b))\ \yLuego \newline 
	(0\ \neq\ minasAdyacentes(t,\ q) \wedge\ \newline
	caminoLibreConBanderitas(t,\ p,\ q,\ b))) \implicaLuego \newline
	posicionPerteneceAJugadas (q,\ j))
}
\bigbreak
\pred{esAdyacente}{p,\ q: \textit{pos}}{
	q[0]\ -\ 1\ \leq\ p[0]\ \leq\ q[0]\ +\ 1 \wedge\ q[1]\ -\ 1\ \leq\ p[1]\ \leq\ q[1]\ +\ 1 \wedge\ p\ \neq\ q
}
\bigbreak
\pred{caminoLibreConBanderitas}{t: \textit{tablero}, \(p_{0}\): \textit{pos}, \(p_{1}\): \textit{pos},\ b: \textit{banderitas}}{
	(\exists s:\ seq<pos>)(caminoLibreSinDefinirMinasAdyacentesALaUltimaPosicionconBanderitas(t,\ p_{0},\ p_{1},\ s,\ b)\ \yLuego\ \newline
	8 > minasAdyacentes(t,\ p_{1}) \geq 1)
}
\bigbreak
\pred{todasLasPosicionesSinMinasAdyacentesConUnaPosicionConfirmadaEnJugadaYCaminoLibreAPertenecenAJugada}{t: \textit{tablero},\ p: \textit{posicion},\ j: \textit{jugada},\ b: \textit{banderitas}}{
	(\forall q: pos)(((posicionPerteneceAlTablero(t,\ q) \wedge \newline
	\neg\ posicionPerteneceABanderitas(q,\ b))\ \yLuego \newline 
	(0\ =\ minasAdyacentes(t,\ q) \wedge\ \newline
	caminoLibreConBanderitas(t,\ p,\ q,\ b) \wedge\ \newline
	existeUnaPosicionDelCaminoLibreQuePerteneceAJugada(t,\ p,\ q,\ j,\ b))) \implicaLuego \newline
	posicionPerteneceAJugadas(q,\ j))
}
\bigbreak
\pred{existeUnaPosicionDelCaminoLibreQuePerteneceAJugada}{t: \textit{tablero},\ p: \textit{posicion}, q: \textit{posicion},\ j: \textit{jugadas},\ b: \textit{banderitas}}{
	(\exists m: pos)((posicionPerteneceAlTablero(t,\ m) \wedge \newline
	posicionPerteneceAJugadas(m,\ j))\ \yLuego \newline 
	(0\ =\ minasAdyacentes(t,\ m) \wedge\ \newline
	caminoLibreConBanderitas(t,\ m,\ p,\ b) \wedge\ \newline
	(\exists s: \TLista{pos})(caminoLibreSinDefinirMinasAdyacentesALaUltimaPosicionConBanderitas(t,\ q,\ m,\ s,\ b))))
}





\subsection{Ejercicio 9}
\bigbreak
\pred{hayPatron121}{t: \textit{tablero}, j: \textit{jugadas}}{
	(\exists s: \TLista{pos})(esPatron121(t,j,s)\ \yLuego\ \newline 
	(\exists p:pos)(posicionPerteneceAlTablero(p,t)\ \yLuego  
	posicionNoPerteneceAJugadas(p,j)\ \wedge\ \newline 
	(\forall i: \ent)(0\ \leq i\ <\ \longitud{s}\ \implicaLuego\ sonAdyacentesNoDiagonales(p, s[i])))
}
\pred{esPatron121}{t: \textit{tablero}, j: \textit{jugadas}, s: \TLista{pos}}{
	posicionesPertenecenATablero(t,s)\ \wedge\ posicionesPertenecenAJugadas(j,s)\ \wedge\ (\longitud{s}\ =\ 3)\ \yLuego\ \newline 
	(\exists\ t: \TLista{pos})(secuenciaOrdenada(t)\ \wedge\ esPermutacion(s,t)\ \yLuego\ cumple121(s));
}
\bigbreak
\pred{posicionNoPerteneceAJugadas}{p: \textit{pos}, j: \textit{jugada}}{
	(\forall i: \ent)(0\ \leq\ i <  \<\ \longitud{j}\ \implicaLuego\ j[i][0] \neq p)
}
\bigbreak 
\pred{sonAdyacentesNoDiagonales}{\(p_{1}\), \(p_{2}\): \textit{pos}}{
	((p_{1}[0]-p_{2}[0])\ =\ 1\ \wedge\ (p_{1}[1]-p_{2}[1])\ =\ 0)\ \vee \newline
	((p_{1}[0]-p_{2}[0])\ =\ 0\ \wedge\ (p_{1}[1]-p_{2}[1])\ =\ 1)\ \vee \newline
	((p_{1}[0]-p_{2}[0])\ =\ -1\ \wedge\ (p_{1}[1]-p_{2}[1])\ =\ 0)\ \vee \newline
	((p_{1}[0]-p_{2}[0])\ =\ 0\ \wedge\ (p_{1}[1]-p_{2}[1])\ =\ -1)
}
\bigbreak
\pred{posicionesPertenecenATablero}{t: \textit{tablero}, s: \TLista{pos}}{
	(\forall i: \ent)(0\ \leq i\ < \longitud{s}\ \implicaLuego\ posicionPerteneceAlTablero(s[i],t))
}
\bigbreak
\pred{posicionesPertenecenAJugadas}{j: \textit{jugadas}, s: \TLista{pos}}{
	(\forall i: \ent)(0\ \leq i\ < \longitud{s}\ \implicaLuego\ posicionPerteneceAJugadas(s[i], j)
}
\bigbreak
\pred{cumple121}{s: \TLista{pos}}{
	((minasAdyacentes(s[0]) = 1)\ \wedge\ (minasAdyacentes(s[1]) = 2)\ \wedge\ (minasAdyacentes(s[2]) = 1))
}
\bigbreak
\pred{secuenciaOrdenada}{\(p_{1}\), \(p_{2}\): \textit{pos}, s: \TLista{pos}}{
	(s[0]\ =\ p_{1}\ \wedge\ s[\longitud{s}\ -\ 1]\ =\ p_{2})\ \wedge \newline
	(\forall i: \ent)(0\ \leq\ i\ <\ \longitud{s} - 1\ \implicaLuego\ esAdyacente(p[i],\ p[i+1]))
}
\bigbreak
\pred{esPermutacion}{s, t: \TLista{pos}}{
	(\longitud{s}\ =\ \longitud{t})\ \wedge\ \newline 
	(\forall x: \textit{pos})(posicionPerteneceASecuencia(p,\ s)\ \iff\ posicionPerteneceASecuencia(p,\ t))
}

\end{document}
